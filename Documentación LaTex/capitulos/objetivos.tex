\chapter{Objetivos}

El objetivo de este proyecto es la creación de un videojuego de terror para PC con la mayor calidad posible compatible con realidad virtual. Para esto se utilizará el motor comercial \textit{Unreal Engine 4}, programando todo lo necesario usando el sistema de \textit{Blueprint visual scripting}. Por tanto, otro objetivo en este \ac{TFG} es el aprendizaje de las diferentes funciones y características del motor \textit{Unreal Engine}.
\\

Además será necesario redactar un \ac{GDD} donde se realizará el diseño previo del videojuego, se analizarán grandes títulos del género para conseguir mayor calidad en el producto final y se crearán los recursos necesarios que serán incluidos en el videojuego, así como modificaciones de los recursos obtenidos de terceros con licencias que me permitan incluirlos en el juego.
\\


\section{Desglose de objetivos}

Estos son los objetivos del \ac{TFG} ordenados y puestos de manera más específica: 

\begin{enumerate}
	\item Realizar el \ac{GDD} del videojuego.
	\item Crear el videojuego base usando el sistema de \textit{blueprints} de \textit{Unreal}.
	\item Crear los modelos necesarios para el videojuego.
	\item Texturizar e incorporar los modelos en el juego.
	\item Importar los recursos de terceros necesarios y que lo permitan según sus licencias.
	\item Completar la lógica del videojuego usando \textit{blueprints}.
	\item Implementar los menús y el \textit{HUD}.
	\item Incorporar sonidos en el juego.
	\item Compatibilizar el juego con el dispositivo de realidad virtual de\textit{ Oculus Rift}.
\end{enumerate}


\begin{comment}


\begin{description}
\item[Índice de contenidos:] (obligatorio siempre) se incluirá un índice de las secciones de las que se componga el documento, la numeración de las 
divisiones y subdivisiones utilizarán cifras arábigas (según UNE 50132:1994) y harán mención a la página del documento donde se ubiquen.
\item[Índice de figuras:] si el documento incluye figuras se podrá incluir también un índice con su relación, indicando la página donde se ubiquen.
\item[Índice de tablas:] en caso de existir en el texto, ídem que el anterior.
\item[Índice de abreviaturas, siglas, símbolos, etc.:] en caso de ser necesarios se podrá incluir cada uno de ellos.
\end{description}
\item[Cuerpo del documento:] en el contenido del documento se da flexibilidad para su organización y se puede estructurar en las secciones que se considere. En todo caso obligatoriamente se deberá, al menos, incluir los siguientes contenidos:
\begin{description}
\item[Introducción:] donde se hará énfasis a la importancia de la temática, su vigencia y actualidad; se planteará el problema a investigar, así como el propósito o finalidad de la investigación.
\item[Marco teórico o Estado del arte:] se hará mención a los elementos conceptuales que sirven de base para la investigación, estudios previos relacionados con el problema planteado, etc.
\item[Objetivos:] se establecerá el objetivo general y los específicos.
\item[Metodología:] se indicará el tipo o tipos de investigación, las técnicas y los procedimientos que serán utilizados para llevarla a cabo; se identificará la población y el tamaño de la muestra así como las técnicas e instrumentos de recolección de datos.
\item[Resultados:] incluirá los resultados de la investigación o trabajo, así como el análisis y la discusión de los mismos.
\end{description}
\item[Conclusiones:] obligatoriamente se incluirá una sección de conclusiones donde se realizará un resumen de los objetivos conseguidos así como de los resultados obtenidos si proceden.
\item[Bibliografía y referencias:] se incluirá también la relación de obras y materiales consultados y empleados en la elaboración de la memoria del \ac{TFG}/\ac{TFM}. La bibliografía y las referencias serán indexadas en orden alfabético (sistema nombre y fecha) o se numerará correlativamente según aparezca (sistema numérico). Se empleará la familia 1 como tipo de letra. Podrá utilizarse cualquier sistema bibliográfico normalizado predominante en la rama de conocimiento, estableciéndose como prioritarios el sistema ISO 690, sistema \ac{APA}  o Harvard (no necesariamente en ese orden de preferencia). En esta plantilla Latex se propone usar el estilo \ac{APA} indicándolo en la línea correspondiente como 
\begin{verbatim}
\bibliographystyle{apalike}
\end{verbatim}


\item[Anexos:] se podrá incluir los anexos que se consideren oportunos.
\end{comment}





