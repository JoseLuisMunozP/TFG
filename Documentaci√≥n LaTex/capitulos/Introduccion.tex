\chapter{Introducción}

Hoy en día la industria del videojuego es uno de los mercados más potentes de nuestra economía. Con el paso de los años este sector se ha ido extendiendo y normalizando hasta llegar al punto en el que se ha convertido en algo que aparece con total normalidad en nuestra vida cotidiana. Hay todo tipo de juegos, desde sencillos juegos de móviles a grandes producciones que no tienen nada que enviar a las grandes películas con presupuestos millonarios de Hollywood, no es de extrañar que mucha gente haya empezado a considerar el videojuego como \textit{el décimo arte.}
\\

Dentro de todos los tipos de videojuegos, hay un género que me interesa especialmente, y es el género del videojuego de terror. Los videojuegos cuentan con una ventaja sobre otros medios audiovisuales como, por ejemplo, el cine, y es que son mucho más inmersivos porque la persona jugando forma parte de la experiencia, no es un mero espectador. Este es un hecho que favorece especialmente al género de terror, y que se acentúa aún más con la llegada de los recientes dispositivos de realidad virtual.
\\

\textbf{"\nombreJuego"} es un videojuego que se aprovecha de esto último, inspirándose en otros grandes juegos del género para alcanzar la mayor calidad posible.
\\


El \textbf{documento} se estructura de la siguiente forma:

\begin{description}
\item[Marco teórico:] En este apartado se estudiará el concepto de videojuego, así como otros videojuegos que son referentes para el mismo.
\item[Objetivos:] En este punto se analizan los diferentes objetivos que se persiguen en la realización de este \ac{TFG}.
\item[Metodología:] Aquí explicaré la metodología que se ha seguido y las herramientas que se han utilizado en la misma.
\item[Cuerpo:] En esta parte se hablará del plan de trabajo seguido, se detallará el \ac{GDD} y todo lo relacionado con el desarrollo del juego, desde las mecánicas hasta su apartado artístico.



\begin{comment}


\begin{description}
\item[Índice de contenidos:] (obligatorio siempre) se incluirá un índice de las secciones de las que se componga el documento, la numeración de las 
divisiones y subdivisiones utilizarán cifras arábigas (según UNE 50132:1994) y harán mención a la página del documento donde se ubiquen.
\item[Índice de figuras:] si el documento incluye figuras se podrá incluir también un índice con su relación, indicando la página donde se ubiquen.
\item[Índice de tablas:] en caso de existir en el texto, ídem que el anterior.
\item[Índice de abreviaturas, siglas, símbolos, etc.:] en caso de ser necesarios se podrá incluir cada uno de ellos.
\end{description}
\item[Cuerpo del documento:] en el contenido del documento se da flexibilidad para su organización y se puede estructurar en las secciones que se considere. En todo caso obligatoriamente se deberá, al menos, incluir los siguientes contenidos:
\begin{description}
\item[Introducción:] donde se hará énfasis a la importancia de la temática, su vigencia y actualidad; se planteará el problema a investigar, así como el propósito o finalidad de la investigación.
\item[Marco teórico o Estado del arte:] se hará mención a los elementos conceptuales que sirven de base para la investigación, estudios previos relacionados con el problema planteado, etc.
\item[Objetivos:] se establecerá el objetivo general y los específicos.
\item[Metodología:] se indicará el tipo o tipos de investigación, las técnicas y los procedimientos que serán utilizados para llevarla a cabo; se identificará la población y el tamaño de la muestra así como las técnicas e instrumentos de recolección de datos.
\item[Resultados:] incluirá los resultados de la investigación o trabajo, así como el análisis y la discusión de los mismos.
\end{description}
\item[Conclusiones:] obligatoriamente se incluirá una sección de conclusiones donde se realizará un resumen de los objetivos conseguidos así como de los resultados obtenidos si proceden.
\item[Bibliografía y referencias:] se incluirá también la relación de obras y materiales consultados y empleados en la elaboración de la memoria del \ac{TFG}/\ac{TFM}. La bibliografía y las referencias serán indexadas en orden alfabético (sistema nombre y fecha) o se numerará correlativamente según aparezca (sistema numérico). Se empleará la familia 1 como tipo de letra. Podrá utilizarse cualquier sistema bibliográfico normalizado predominante en la rama de conocimiento, estableciéndose como prioritarios el sistema ISO 690, sistema \ac{APA}  o Harvard (no necesariamente en ese orden de preferencia). En esta plantilla Latex se propone usar el estilo \ac{APA} indicándolo en la línea correspondiente como 
\begin{verbatim}
\bibliographystyle{apalike}
\end{verbatim}


\item[Anexos:] se podrá incluir los anexos que se consideren oportunos.
\end{comment}
\end{description}

\textbf{REVISAR CUANDO SE HAYA TERMINADO EL DOCUMENTO}


